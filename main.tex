\documentclass{article}
\usepackage[utf8]{inputenc}
\usepackage{mdframed}
\usepackage{subfiles}
\usepackage{subcaption}
\usepackage{graphicx}
\usepackage{wrapfig}
\usepackage{amsmath}
\usepackage{amsfonts}
\usepackage{polski}

\newmdtheoremenv{tw}{Twierdzenie}
\newmdtheoremenv{stw}{Stwierdzenie}
\newtheorem{obserwacja}{Obserwacja}
\newtheorem{dowod}{Dowód}
\newmdtheoremenv{definicja}{Definicja}
\newtheorem{przyklad}{Przykład}
\newtheorem{pytanie}{Pytanie}



\title{Notatki z Analizy II L2019, FUW}
\author{Jakub Korsak}
\begin{document}
\maketitle

\pagebreak
\section{Wykład (26.02.2019)}
\subfile{tex/wyklad1.tex}
\pagebreak
\section{Wykład (01.03.2019)}
\subfile{tex/wyklad2.tex}
\pagebreak
\section{Wykład (05.03.2019)}
\subfile{tex/wyklad3.tex}
\pagebreak
\section{Wykład (08.03.2019)}
\subfile{tex/wyklad4.tex}
\pagebreak
\section{Wykład (12.03.2019)}
\subfile{tex/wyklad5.tex}
\pagebreak
\section{Wykład (15.03.2019)}
\subfile{tex/wyklad6.tex}
\pagebreak
\section{Wykład (19.03.2019)}
\subfile{tex/wyklad7.tex}
\pagebreak
\section{Wykład (22.03.2019)}
\subfile{tex/wyklad8.tex}
\pagebreak
\section{Wykład (26.03.2019)}
\subfile{tex/wyklad9.tex}
\pagebreak




\end{document}
